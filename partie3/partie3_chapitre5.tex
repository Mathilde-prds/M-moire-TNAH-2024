\chapter{Fermeture du deuxième diamant : développement et diffusion de la fonctionnalité}
\subsection{Spécifications fonctionnelles détaillées}

Afin de s’assurer que le développeur possède bien l’ensemble des informations nécessaires pour développer ce qui a été décidé par le précédent travail, le \textit{\gls{Product Manager}} et le \textit{\gls{Product Owner}} d’\gls{Archifiltre} écrivent les spécifications fonctionnelles détaillées. Ces dernières permettent de décrire en détail l’ensemble du comportement de chaque interface du processus. Il faut ainsi décrire par exemple l’effet produit par chaque bouton. Par ailleurs, l’équipe se sert du moment de sa rédaction également pour approfondir la réflexion sur les erreurs pouvant potentiellement se produire à chaque étape du processus. Elles sont listées dans les spécifications associées au message d’erreur à afficher pour chacune d'elles. Enfin, le développeur ayant suivi et parfois participé au travail effectué, la définition des spécifications se base également sur l’expertise technique qu’il a pu transmettre à l’équipe tout au long des choix faits lors des étapes précédentes permettant ainsi de prendre en compte les complexités techniques dans ces choix et d'éviter de mauvaises surprises plus tard comme par exemple des temps importants de développement supplémentaires. 


Traditionnellement, les spécifications prennent la forme d’une partie du cahier des charges et sont rédigées de manière linéaire, parfois appuyées d’images sans que cela ne soit systématique. Il en existe deux sortes : les spécifications fonctionnelles générales et les spécifications fonctionnelles détaillées. La première correspond dans la méthode \gls{Archifiltre} au \gls{PRD} et permet de faire l’étude générale du besoin alors que la deuxième permet de détailler la solution afin de guider le développement. \gls{Archifiltre} fait le choix d’une forme plus visuelle afin de rendre plus claires et instinctives ces spécifications fonctionnelles détaillées\footnote{Pour aller plus loin concernant l’amélioration des spécifications : \cite{cagan_revisiting_2006}}. Les \textit{\gls{wireframe}s} validés sont associés aux spécifications et chaque précision est liée à son emplacement sur le \textit{\gls{wireframe}} à l’aide de flèche comme ci-dessous : 

\insererImage{0.5}{illustrations/figure26.png}{Exemple de spécifications fonctionnelles détaillées associées au \gls{wireframe} de l'interface correspondante.\protect\footnotemark}{figure26}
\footnotetext{Zoom disponible en annexe \ref{annexe10}}

A partir de ces spécifications, le développeur va pouvoir organiser son travail et construire le produit attendu. Il est important de noter que ces dernières évolueront nécessairement, bien que le but lors de leur rédaction soit de les limiter autant que possible, en fonction des solutions techniques proposées par le développeur et des blocages qui peuvent apparaître. Des réunions à fréquence régulière sont donc organisées entre le développeur et le \textit{\gls{Product Manager}} afin de valider les choix techniques et leur impact sur ce qui a été défini dans les spécifications. 

\subsection{Adoption de la nouvelle fonctionnalité}


Lorsque les développements du \gls{MVP} seront terminés, un des enjeux essentiels de cette fonctionnalité devra alors être mené : son adoption par les services d’archives. Comme nous avons pu l’expliquer précédemment, la particularité de cette fonctionnalité est le fait que peu de services d’archives, et donc d’utilisateurs, se sont déjà saisis de la problématique à laquelle elle répond. Ainsi, il s’agit d’une fonctionnalité assez ambitieuse dont la volonté est de créer une méthodologie de traitement des systèmes d’information à archiver sans utiliser de connecteurs. Pour atteindre cet objectif, il est nécessaire de prévoir une stratégie de communication efficace afin de faciliter la prise en main de cette fonctionnalité relativement complexe. 


La première étape pour ce faire s’est mise en place lors des réflexions autour de la définition de l’interface utilisateur. En effet, un des enjeux était de rendre ce processus complexe le plus intuitif et clair possible. Pour ce faire, la notion de jointure a par exemple été rendue invisible pour l’utilisateur car les tests ont révélé une complexité, jugée finalement non nécessaire par l’équipe, à la compréhension de ce concept par les utilisateurs. Le choix final détaillé précédemment de l’interface d’association de balises \gls{SEDA} aux métadonnées qui propose la définition de chaque balise issue du dictionnaire du \gls{SEDA} 2.2 est également pensé afin de réduire la complexité du \gls{SEDA} qui repousse une partie des archivistes. Enfin, une documentation accompagnant la prise en main de la fonctionnalité par l’utilisateur va être proposée. Sa forme n’est pour le moment pas définie mais parmi les options envisagées se trouve la possibilité de fournir une arborescence test et un CSV de métadonnées associées ainsi qu’un pas à pas correspondant afin d’accompagner l’utilisateur dans sa première expérience de la fonctionnalité. Dans le but de créer la meilleure documentation possible, des entretiens utilisateurs pourront être organisés selon les principes de la méthode de travail générale de l’équipe \gls{Archifiltre}.  


La stratégie d’adoption de cette fonctionnalité imaginée par \gls{Archifiltre} consiste également à tester directement au sein du service des archives des Ministères sociaux cette solution et de documenter ce test. Ainsi, cette expérience pourra par la suite être partagée lors de présentations \gls{Archifiltre} et servir d’exemple. La méthode de traitement ainsi illustrée concrètement pourrait alors servir de base à l’appropriation de cette fonctionnalité par d’autres services d’archives. Par ailleurs, le fait d’échanger régulièrement avec les utilisateurs tout au long de ce développement a également permis d’impliquer des utilisateurs qui se feront potentiellement par la suite ambassadeurs de cette fonctionnalité au sein de leurs services.


Enfin, une communication par mail ainsi que sur les réseaux sociaux, notamment \textit{LinkedIn}, de manière plus classique sera également réalisée, s’intégrant dans le plan de communication général actuellement mis en place pour \gls{Archifiltre}. Cette communication permettra notamment dans un second temps d’obtenir des retours utilisateurs et de mesurer l’utilisation de cette fonctionnalité. Un plan de \textit{tracking}, qui sera mis en place lors du développement de la fonctionnalité, fournira également des statistiques complémentaires à ces retours et permettra d’évaluer sa conformité aux objectifs quantitatifs définis dans le \gls{PRD}. Ceux-ci seront décidés lors de la mise en place du \textit{tracking} mais se présentent comme des données chiffrées souhaitées sur une période donnée, comme par exemple le nombre d’utilisateurs ou le nombre de fichiers de métadonnées importés au premier mois. Cet ensemble permettra à l’équipe d’alimenter la \textit{\gls{story map}} et de continuer l’évolution et la correction de bugs de cette fonctionnalité dans de futures versions. 
\\

Ainsi, bien que le travail de cadrage de cette fonctionnalité soit terminé, il demeure deux parties essentielles de sa création à réaliser : le développement et la communication afin d’assurer son adoption puis d’en mesurer l’impact. Par ailleurs, l’ensemble de la méthodologie décrite dans cette troisième partie et présentée sur la base d’un exemple précis se fait ici l'exemple du déploiement de la stratégie innovante de gestion de projet d’\gls{Archifiltre}. 
