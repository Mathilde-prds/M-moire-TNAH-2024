\newglossaryentry{OAIS}{
	name={OAIS},
	description=({\textit{Open Archival Information System}). Modèle de référence pour un système d'archivage ouvert et informatisé}
}

\newglossaryentry{SEDA}{
	name={SEDA},
	description={(Standard d'Echange de Données Archivistiques). Norme française qui définit un format d'échange pour les archives électroniques entre les systèmes producteurs, les systèmes d'archivage et les systèmes d'accès. Le SEDA est utilisé pour garantir l'interopérabilité entre les différentes plateformes de gestion des archives}
}

\newglossaryentry{VITAM}{
	name={VITAM},
	description={(Valeurs immatérielles transmises aux archives pour mémoire). Programme interministériel français d'archivage électronique. Il propose une solution \textit{open source } pour la gestion, la conservation et l'accès aux archives électroniques à long terme, en conformité avec les standards et les bonnes pratiques archivistiques. \href{https://www.programmevitam.fr/}{https://www.programmevitam.fr/}}
}

\newglossaryentry{Product Owner}{
	name={Product Owner},
	description={Chef de produit digital en mode agile. Il est  responsable de la définition et de la conception d'un produit selon la méthodologie agile}
}

\newglossaryentry{Product Manager}{
	name={Product Manager},
	description={Chargé(e) d'évaluer les opportunités et de déterminer ce qui sera construit et livré aux utilisateurs}
}

\newglossaryentry{agile}{
	name={agile},
	description={Approche de gestion de projet qui privilégie l'adaptabilité, la collaboration et l'itération rapide. La méthode Agile favorise la livraison continue de produits ou de services à travers de courtes périodes de développement appelées \enquote{sprints}, permettant de répondre plus efficacement aux changements et aux besoins des clients}
}

\newglossaryentry{wireframe}{
	name={wireframe},
	description={Schéma simplifié représentant l'interface d'une application ou d'un site web, utilisé pour planifier la structure, le contenu et la navigation. Les \textit{wireframes} sont des outils de conception visuelle qui permettent de définir l'agencement des éléments avant de passer à la phase de design détaillé}
}

\newglossaryentry{story map}{
	name={story map},
	description={Outil visuel utilisé dans le cadre des méthodes agiles pour représenter les fonctionnalités d'un produit ou d'un projet sous forme de récits utilisateurs (\enquote{\textit{user stories}}). Une \textit{storymap} aide à organiser et à prioriser les tâches en fonction de la valeur qu'elles apportent aux utilisateurs, tout en fournissant une vue d'ensemble du projet}
}

\newglossaryentry{SIP}{
	name={SIP},
	description={(\textit{Submission Information package}). Forme entrante du paquet de données dans le SAE. Il est généré par le producteur selon le modèle imposé par le gestionnaire du SAE}
}

\newglossaryentry{AIP}{
	name={AIP},
	description={(\textit{Archival Information Package}). Forme que prennent les données pour être gérées spécifiquement par le gestionnaire. L'AIP est conçu pour contenir tout ce qui est nécessaire pour comprendre, authentifier et rendre utilisable l'information archivée sur une longue période, même si le contexte technologique change}
}

\newglossaryentry{DIP}{
	name={DIP},
	description={(\textit{Dissemination Information Package}). Forme que prennent les données en sortie du SAE afin d’être consultées par un utilisateur selon sa requête, ses droits et les droits de diffusion de l’information requêtée}
}


\newglossaryentry{Archifiltre}{
	name={Archifiltre},
	description={Startup d'Etat à l'origine d'une application \textit{open source} permettant la visualisation d’arborescences complètes afin de pouvoir les appréhender rapidement en vue de les décrire, les organiser, les trier et les enrichir en apportant de la contextualisation et de la qualification aux documents.}
}

\newglossaryentry{EIG}{
	name={EIG},
	description={(Entrepreneurs d'ingtérêt général). Programme du gouvernement français qui recrute des experts en numérique pour travailler sur des projets d'innovation au sein de l'administration, afin de répondre à des défis d'intérêt public}
}

\newglossaryentry{SI}{
	name={SI},
	description={(Système d'information). Composé de l'ensemble des infrastructures et services logiciels informatiques permettant de collecter, traiter, transmettre et stocker les données sous forme numérique qui concourent aux missions des services de la structure}
}

\newglossaryentry{approche produit}{
	name={approche produit},
	description={Méthodologie centrée sur la création, le développement et l'amélioration continue d'un produit spécifique. L'approche produit se concentre sur la satisfaction des besoins des utilisateurs, la vision du produit, et l'alignement des objectifs de l'équipe avec les exigences du marché}
}

\newglossaryentry{SAE}{
	name={SAE},
	description={(Système d'archivage électronique). Logiciel ou plateforme permettant de stocker, gérer, et conserver des documents numériques de manière sécurisée et conforme aux normes, facilitant leur recherche et leur exploitation à long terme}
}

\newglossaryentry{UX}{
	name={UX},
	description={(\textit{User Experience}). Ensemble des aspects relatifs à l'expérience globale ressentie par un utilisateur lors de l'utilisation d'un produit, d'un service ou d'une interface numérique, incluant la facilité d'utilisation, la satisfaction et l'émotion}
}

\newglossaryentry{UI}{
	name={UI},
	description={(\textit{User Interface}). Partie visible d'un produit ou service numérique avec laquelle l'utilisateur interagit, comprenant les éléments graphiques, les boutons, les menus, et la navigation}
}

\newglossaryentry{benchmark}{
	name={benchmark},
	description={Processus d'évaluation des performances d'un produit, service ou processus en le comparant à des standards ou des concurrents, afin d'identifier des meilleures pratiques ou des axes d'amélioration}
}

\newglossaryentry{PRD}{
	name={PRD},
	description={(\textit{Product Requierement Document}). Document détaillant les spécifications fonctionnelles, techniques, et les exigences d'un produit à développer, servant de référence pour les équipes de conception et de développement}
}

\newglossaryentry{XML}{
	name={XML},
	description={(\textit{eXtensible Markup Language}). Langage de balisage flexible permettant de structurer, stocker, et transporter des données de manière lisible par les humains et les machines, largement utilisé pour les échanges de données sur le web}
}

\newglossaryentry{ETL}{
	name={ETL},
	description={(\textit{Extract, Transform, Load}). Processus utilisé en informatique pour extraire des données de différentes sources, les transformer selon des règles spécifiques, et les charger dans un système de stockage, comme un entrepôt de données}
}

\newglossaryentry{mapping}{
	name={mapping},
	description={Processus de correspondance entre deux jeux de données ou modèles, permettant de transformer ou transférer des informations d'un format ou d'une structure à une autre}
}

\newglossaryentry{MVP}{
	name={MVP},
	description={(\textit{Minimum Viable Product}). Version la plus simple et fonctionnelle d'un produit qui permet de tester une idée ou une fonctionnalité avec un minimum de ressources, afin de recueillir des retours utilisateurs avant de poursuivre son développement}
}

\newglossaryentry{manifest}{
	name={manifest},
	description={Document ou fichier structuré contenant les métadonnées et la description des contenus numériques inclus dans un SIP, utilisé pour le transfert de ces contenus vers un système d'archivage électronique}
}

\newglossaryentry{UML}{
	name={langage UML},
	description={(\textit{Unified Modeling Language}). Langage de modélisation visuelle standard utilisé pour concevoir, visualiser, et documenter les systèmes logiciels, en utilisant divers types de diagrammes (comme les diagrammes de classes ou de cas d'utilisation)}
}

\newglossaryentry{front-office}{
	name={front-office},
	description={Ensemble des interfaces et des services d'un système ou d'une application avec lesquels les utilisateurs interagissent directement, souvent associé aux aspects \textit{client-facing}}
}

\newglossaryentry{back-office}{
	name={back-office},
	description={Partie d'un système ou d'une application qui n'est pas visible par les utilisateurs finaux, mais qui gère les processus internes, les bases de données, et le support administratif ou technique}
}


\newglossaryentry{DTD}{
	name={DTD},
	description={(\textit{Document Type Definition}) : Spécification définissant la structure et les règles de validation des documents XML, indiquant quels éléments et attributs peuvent apparaître dans un document, et comment ils doivent être organisés}
}


\newglossaryentry{EAD}{
	name={EAD},
	description={(\textit{Encoded Archival Description}). Norme XML pour décrire des collections d'archives, facilitant l'accès et la gestion des descriptions archivistiques au format numérique}
}

\newglossaryentry{XSD}{
	name={XSD},
	description={(\textit{XML Schema Definition}).  Langage de schéma utilisé pour définir la structure et les contraintes des documents XML, offrant un moyen de valider la conformité d'un document XML par rapport à un modèle défini}
}

\newglossaryentry{ISAD(G)}{
	name={ISAD(G)},
	description={(\textit{International Standard Archival Description - General}). Norme internationale pour la description archivistique, fournissant des règles et des lignes directrices pour créer des descriptions cohérentes et compréhensibles de documents d'archives}
}

\newglossaryentry{unité d’archives}{
	name={unité d’archive},
	description={(ou Archive Unit). Ensemble cohérent de documents ou d'objets archivistiques qui sont regroupés et conservés en tant qu'entité unique, souvent en raison de leur provenance ou de leur pertinence commune}
}

\newglossaryentry{POC}{
	name={POC},
	description={(\textit{Proof of concept}). Prototype ou démonstration réalisée pour valider la faisabilité technique d'une idée ou d'une solution, avant de s'engager dans un développement complet ou une mise en production}
}