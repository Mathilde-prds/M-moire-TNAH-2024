\chapter{L'appropriation française : le SAE VITAM et l’outil ReSIP}
Définissant conceptuellement le fonctionnement d’un système d’archivage électronique, le modèle \gls{OAIS} se diffuse à l’international et notamment en France où on se l’approprie en développant des outils suivant ses concepts.

 
\subsection{Les débuts de l’archivistique numérique française}
Dès 2005, les premières réflexions sur la mise en place d’un système d’archivage électronique à l’échelle nationale sont mises en place. Le projet associé, nommé PIL@E, devait suivre les recommandations du référentiel général d’interopérabilité (RGI)\footnote{Le RGI est un \enquote{cadre de recommandations référençant des normes et standards qui favorisent l’interopérabilité au sein des systèmes d’information de l’administration} (\cite[p.64]{gueit-montchal_chapitre_2020})
}, c’est-à-dire correspondre non seulement au modèle \gls{OAIS} mais aussi aux formats et à la politique d’archivage conseillés. À l’heure actuelle, c’est en effet le référentiel général d’interopérabilité\footcite{noauthor_referentiel_nodate}, porté par la direction interministérielle du numérique (DINUM) qui propose pour l’archivage numérique un profil d’interopérabilité.


En parallèle, le Service interministériel des archives de France (SIAF) et ce qui était la Direction générale de la modernisation de l’État (DGME) ont travaillé au développement d’un format de métadonnées adapté à la simplification des échanges entre services producteurs/versants et services d’archives : le Standard d’échange de données pour l’archivage ou \gls{SEDA}, sur lequel nous reviendrons dans la deuxième partie de ce travail et qui s’inscrit dans ces grands travaux de normalisation et de définition d’un cadre à l’archivage électronique en France. En 2006, la même année que la parution de la première version du \gls{SEDA}, le projet de pilote national d’archivage électronique PIL@E est lancé. Son but était de permettre la réutilisation du modèle et des outils créés lors de ce projet par les services d’archives ou les services producteurs désirant développer leur propre plate-forme\footcite{sibille-de_grimouard_projet_2015}.


Bien qu’ayant été abandonné en avril 2011 pour diverses raisons que nous ne détaillerons pas ici, ce projet s’est positionné d’après Claire Sibille-de Grimoüard, alors sous-directrice de la politique archivistique au SIAF, comme  \enquote{une expérience fondatrice dans le domaine de l’archivage numérique}, en encourageant l'accélération de la mise en place de projets d’archivage électronique dans les collectivités territoriales et en donnant un espace à la première implémentation de ce qui sera le standard d’échange de données, le \gls{SEDA}.

\subsection{Le Programme VITAM}
A partir de 2011, les premières réflexions autour de ce qui deviendra le Programme \gls{VITAM} sont menées. S’inspirant des connaissances et de l’expérience acquises par le biais de PIL@E, le programme \gls{VITAM} est officiellement lancé en 2015. Suivant les mêmes objectifs de réutilisation à l’échelle nationale, d’adaptation à chaque institution et d’implémentation du \gls{SEDA}, \gls{VITAM} commence d’abord sous la forme d’un \textit{\gls{back-office}} seul, sans interface utilisateur (ou \textit{\gls{front-office}}). Ce choix s’explique par une divergence d’opinion concernant le développement de cette dernière entre les trois ministères porteurs du programme, le Ministère de la Culture, le Ministère des Affaires étrangères et le Ministère des Armées, qui ont donc choisi de ne mutualiser que le \textit{\gls{back-office}}. Le programme prend seulement en charge le stockage technique de l’information. Son fonctionnement est résumé par le programme \gls{VITAM} au moyen du schéma suivant\footcite{noauthor_focus_nodate} :

\clearpage 


\insererImage{0.4}{illustrations/figure4.png}{Schéma du fonctionnement de \gls{VITAM} d’après le modèle \gls{OAIS}}{figure4}


On retrouve dans ce fonctionnement les six entités fonctionnelles principales qui structurent le modèle \gls{OAIS} : entrées (\textit{Ingest}), stockage (\textit{Archival Storage}), gestion de la donnée (\textit{Data Management}), planification de la pérennisation (\textit{Preservation Planning}), administration (\textit{Administration}) et accès (\textit{Access}). Chacune répond à un rôle, comme décrit précédemment, et structure le fonctionnement du \gls{SAE}. 


Les trois ministères associés pour son développement prennent alors chacun en charge le développement de leur propre interface utilisateur au \textit{\gls{back-office}} \gls{VITAM}, connues sous les noms ADAMANT (Accès et Diffusion des Archives et des Métadonnées des Archives Nationales dans le Temps) pour le Ministère de la Culture, Saphir pour le Ministère des Affaires étrangères et Archipel pour le Ministère des Armées\footcite{noauthor_et_nodate}.


Lors de sa présentation devant les missions archives des différents Ministères, les équipes de \gls{VITAM} se sont servies d’un logiciel développé par le directeur du programme, Jean-Séverin Lair, afin de mettre sous la forme d’un \gls{SIP} les archives électroniques versées dans le \gls{SAE} \gls{VITAM}\footnote{Entretien avec Edouard Vasseur (21 juin 2024)}. Cet outil allait devenir ReSIP, pour \enquote{Réaliser et Editer des \gls{SIP}}, application gérée par le programme \gls{VITAM} permettant de \enquote{construire et manipuler des structures arborescentes d’archives, d’en éditer les métadonnées, de les importer et exporter sous la forme de \gls{SIP}, sous la forme de hiérarchie disque ou encore sous forme CSV (pour \textit{comma separated values}) pour les plans de classement}\footcite{noauthor_resip_nodate}. ReSIP génère donc des \gls{SIP}. Ces derniers prennent la forme d’un zip au sein duquel les fichiers (aussi appelés binaires) sont regroupés dans un dossier et renommés avec des identifiants (ID) en série continue (ID01, ID02, etc). Au même niveau que ce dossier se trouve un document en \gls{XML} \gls{SEDA}, standard indispensable pour le versement et le traitement des archives au sein du \gls{SAE} \gls{VITAM}, nommé le \textit{\gls{manifest}}. Ce dernier contient toutes les métadonnées associées aux fichiers et au versement dans sa globalité.


Par ailleurs, ReSIP est construit à partir de la bibliothèque \textit{sedalib} qui permet la manipulation de paquet \gls{SEDA} et qui fait partie du projet \textit{sedatools} porté par l’équipe du Programme \gls{VITAM} et le SIAF. Celui-ci est accessible sur le Github du SIAF\footcite{noauthor_programmevitamsedatools_2024} et est constitué \enquote{d’outils utiles aux développeurs et testeurs pour la construction et manipulation des \gls{SIP} conformes au \gls{SEDA}}. Il est composé de six modules dont la bibliothèque \textit{sedalib} et le logiciel ReSIP. Le but de ce projet est de rendre la création de paquets conforme au \gls{SEDA} utilisable par le plus grand nombre.


Par la suite, ReSIP devient nécessaire au versement dans \gls{VITAM} et s’intègre comme étape de la chaîne de traitement et de versement des archives numériques alors en pleine construction.

\subsection{Élargissement du Programme VITAM : VITAM UI et VaS}

\gls{VITAM} reste néanmoins encore limité en termes d’utilisation à ce stade, notamment en raison de l’investissement que son adoption demande. Pour l’utiliser, il faut non seulement développer sa propre interface compatible au \textit{\gls{back-office}} \gls{VITAM} mais également acquérir ses propres infrastructures de stockage et les maintenir. Dans ce cadre, en 2019, le CEA (Commissariat à l'énergie atomique et aux énergies alternatives) et le CINES (Centre informatique national de l'enseignement supérieur), rapidement rejoints par Locarchives - Xelians, lancent l’initiative VITAM UI (pour interface utilisateurs). Cette initiative se matérialise sous la forme d’un entrepôt de code permettant de constituer une interface. Cet entrepôt s’organise en paquets de codes séparés pour chaque fonctionnalité du \textit{backlog} de \gls{VITAM}. VITAM UI a ensuite été ouvert aux membres du Club utilisateurs \gls{VITAM} intéressés et aux acteurs du projet \gls{VITAM} accessible en service (VaS) liés par un accord définissant les règles de contribution\footcite{noauthor_vitam_nodate}.


Aujourd’hui, le CEA, le CINES, Locarchives-Xelians et VaS travaillent chacun de leur côté pour développer les fonctionnalités qu’ils se sont réparties avant de les mettre en commun. Ce travail s’adapte alors à la partie \textit{\gls{back-office}} de \gls{VITAM} et les nouveautés sont livrées en même temps. L’équipe \gls{VITAM} reste le décisionnaire prioritaire pour ce projet et possède un droit de veto \enquote{au titre de son engagement dans la durée pour la maintenabilité et la correction de tous les outils \gls{VITAM}}. Un de ses représentants exerce ce droit au sein des comités de cohérence et comité technique mis en place pour assurer la gouvernance de VITAM UI et auxquels sont associés un représentant de chaque projet contributeur, dont VaS\footcite{noauthor_vitam_nodate}.


Le projet VITAM accessible en service (VaS) s’est quant à lui développé en parallèle de VITAM UI à partir de 2019 notamment afin de répondre aux besoins de gestion de l’archivage intermédiaire. Cette extension au Programme \gls{VITAM} a fait intervenir plusieurs contributeurs illustrés par le Programme \gls{VITAM} dans le schéma suivant\footcite{noauthor_vitam_nodate-1} : 


\insererImage{0.4}{illustrations/figure5.png}{Schéma des acteurs de la comitologie de VaS}{figure5}



Le projet VaS est d’abord lancé avec ces acteurs, puis le projet est rejoint par de nouveaux Ministères avant d’être ouvert plus largement à de multiples organisations publiques. Il s'appuie sur la réutilisation de la solution logicielle \gls{VITAM} à laquelle s’ajoute l’infrastructure interministérielle mutualisée avec un \textit{\gls{front-office}} générique, VITAM UI, et la prise en charge des infrastructures de stockage, hébergées par le Ministère des Finances. Ce stockage est une solution dite SaaS (\textit{Software as a Service}) c'est-à-dire \enquote{un logiciel en tant que service}. L’utilisation d’un logiciel SaaS permet de ne pas avoir à stocker ses données en interne grâce à un hébergement informatique sur des serveurs externes dits \textit{cloud}. \enquote{Par ailleurs, la solution est mise à jour régulièrement avec des améliorations continues et de nouvelles fonctionnalités, sans maintenance à effectuer de la part des utilisateurs. Il n’y a pas besoin d’avoir un service informatique interne puisque c’est le fournisseur du service qui en est responsable}\footcite{pascaud_pourquoi_nodate}.


VaS est aujourd’hui une solution payante qui repose sur le principe de répartition des coûts entre tous les utilisateurs. Ce service continue aujourd’hui d’évoluer en lien avec les acteurs porteurs du projet. De nouvelles fonctionnalités sont régulièrement ajoutées et un des objectifs à terme est de permettre le transfert des archives définitives dans la plate-forme des Archives nationales. Il s’agit finalement d’un modèle de \gls{VITAM} particulièrement répandu dans les Ministères mais souvent pour des besoins ciblés. Les Ministères sociaux l'utilisent par exemple uniquement pour l’archivage des boîtes mails pour le moment alors que le Ministère de l’Agriculture, lui, s’en sert pour l’archivage intermédiaire des documents bureautiques. Ces utilisations évolueront très certainement dans le futur avec la poursuite des développements des fonctionnalités offertes par VaS.
\insererImage{0.6}{illustrations/figure6.png}{Schéma récapitulatif de l’environnement \gls{VITAM}}{figure6}
