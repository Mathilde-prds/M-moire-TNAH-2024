\chapter{Le cas de l’archivage des systèmes d’information}

\subsection{Les nouveaux objets électroniques à archiver}
A l’aide de la chaîne de traitement telle que décrite précédemment, une méthodologie pour l’archivage des fichiers bureautiques et audiovisuels a pu être établie. Grâce à elle, les données de ces documents sont facilement converties en \gls{SEDA} dans des \gls{SIP}. Néanmoins, bien que représentant la majorité des informations électroniques à archiver, les fichiers bureautiques et audiovisuels n’en constituent pas la totalité. Certaines typologies plus complexes demandent une attention particulière et surtout une adaptation des méthodes et des outils\footcite[p.95]{dangio-barros_archivage_2013}. En effet, aujourd'hui des objets numériques complexes comme les mails, au format .pst le plus souvent dans l’administration française, attirent l’attention des spécialistes et des solutions sont encore en train d’être réfléchies\footnote{Pour en savoir plus, voir :\cite{programme_vitam_larchivage_2013}}.


Depuis le programme \textit{Action publique 2022} (AP2022) porté par le Premier ministre en 2017 et visant à la dématérialisation de l’ensemble des processus administratifs, les projets de systèmes d’information nationaux\footnote{D’après le décret n° 2014-879 du 1er août 2014 relatif au système d'information et de communication de l'Etat, un \enquote{système d'information et de communication de l'Etat est composé de l'ensemble des infrastructures et services logiciels informatiques permettant de collecter, traiter, transmettre et stocker les données sous forme numérique qui concourent aux missions des services de l'Etat.} <\href{https://www.legifrance.gouv.fr/loda/id/JORFTEXT000029337021}{https://www.legifrance.gouv.fr/loda/id/JORFTEXT000029337021}>} se sont renforcés et multipliés. Outre le changement de supports et d’acteurs agissant sur ces archives induit par un tel bouleversement, cela a également renforcé le besoin de maîtriser les données des systèmes d’information afin de les archiver efficacement\footcite{banat-berger_note_2023}. Or, les données que ces systèmes contiennent peuvent être multiples, sous forme de documents déposés, de bases de données ou autre. 


Dans une \textit{Note sur l'archivage centralisé des données, documents numériques et métadonnées issus des systèmes d'information nationaux de l'État} datant du 03 août 2023, Françoise Banat-Berger, cheffe de service du Service interministériel des archives de France (SIAF), affirme la responsabilité de la Mission des archives de France et des services d’archives ministériels dans le \enquote{traitement et le transfert} des \enquote{données, documents et métadonnées} des systèmes d’information nationaux \enquote{vers les Archives nationales, en vue de leur archivage définitif}. Pour faire face à cette mission, le Ministère de l’Agriculture a notamment créé une grille d’étude des systèmes d’information appelée \enquote{étude CYCLAD}\footnote{Voir la grille d'étude CYCLAD en annexe \ref{annexed}} (Cycle de conservation légal et d'archivage des données). Cette grille est également utilisée par les Ministères sociaux et permet aux archivistes de faire un état des lieux des données présentes dans les systèmes d’information, de leur importance historique et juridique et de leurs durées de conservation afin de définir en amont une stratégie efficace d’archivage. 


Par ailleurs, il faut noter que la définition concrète de cet enjeu dans l’objectif 8.2 \enquote{Archiver au niveau central les données des services déconcentrés de l’Etat issues d’applications développées et maintenues au niveau central} du \textit{Cadre commun stratégique de modernisation des archives} pour la période 2020-2024\footcite{noauthor_cadre_nodate} appuie également l’importance que la maîtrise de l’archivage des systèmes d’information va continuer de prendre dans les années futures.

\subsection{La naissance d’un besoin au travers de l’expérience des Ministères sociaux}
\label{part2.chap2.2}

Au bureau des archives des Ministères sociaux, porteurs de l’outil \gls{Archifiltre}, la problématique de l’archivage des systèmes d’information (\gls{SI}) s’est imposée d'elle-même au fur et à mesure des cas d’usage. C’est en 2019 que le service a traité son premier cas d’archivage de système d’information avec le \gls{SI} Papyrus gérant l’ensemble des rapports de l’Inspection générale des affaires sociales (IGAS). Ce premier cas était peu complexe car il ne comportait qu’une base de données à archiver. 


En revanche, Chloé Moser, adjointe à la cheffe de bureau et \textit{Product owner} d’\gls{Archifiltre}, avait pu suivre, au titre du contrôle scientifique et technique, en 2019 le décommissionnement d’un \gls{SI} de l’Agence nationale de sécurité sanitaire de l'alimentation, de l'environnement et du travail (ANSES) qui possédait un niveau de complexité supérieur. La volonté était de récupérer les métadonnées du \gls{SI} pour enrichir les documents versés dans le \gls{SAE} de l’ANSES. Pour mener à bien cet objectif, c’est le prestataire Mintika qui a effectué les exports. Dès lors, au sein du service est née une réflexion sur le manque d’autonomie qu’une telle solution crée. Il a semblé alors nécessaire de trouver un moyen pour pouvoir effectuer les exports sans avoir besoin de faire appel à un prestataire pour chaque \gls{SI}.


En 2021, le cas du système d’information CONTIX+ marque un tournant dans les réflexions et confirme le besoin d’un outil qui permette l’archivage des \gls{SI} complexes. CONTIX+ est un \gls{SI} de la Direction des affaires juridiques (DAJ) des Ministères sociaux qui permet le suivi des contentieux. Son décommissionnement a été décidé afin de le remplacer à la fois par un espace \textit{Sharepoint} et par la plateforme télérecours du Conseil d’Etat. La complexité de son archivage réside dans le fait qu’il ne s’agisse pas simplement d’une base de données mais à la fois de documents et de métadonnées associées. Ainsi, lors de l’export du \gls{SI} pour archivage nous nous retrouvons avec une arborescence de documents d’un côté et un fichier CSV de l’autre. Ce fichier CSV est constitué d’autant de colonnes qu’il y a de catégories de métadonnées à extraire de ce \gls{SI} et autant de lignes qu’il y a de documents extraits. Ainsi, les documents ne sont associés qu’aux métadonnées minimum classiquement attachées à des documents comme la date de création, le titre ou le format et il est nécessaire de trouver la ligne du CSV qui leur correspond afin de connaître des informations supplémentaires entrées dans le \gls{SI} et les concernant. L’archivage de ce \gls{SI} s’avérant être un bon exemple des problématiques auxquelles les archivistes vont être de plus en plus confrontés dans un futur relativement proche, le service des archives des Ministères sociaux a fait appel au prestataire Olkoa grâce au financement obtenu par une prestation DIAMAN\footcite{noauthor_appel_2024} (Dispositif d'accompagnement des missions pour l'archivage numérique). Avec leur aide, un plan d’archivage de CONTIX+ a pu être mis en place et les premières réflexions sur le moyen de faire évoluer l’outil Archifiltre afin de rendre l’archivage des \gls{SI} moins complexe ont été menées. Le diagramme suivant présente ce plan d'archivage. Il est constitué de trois étapes : l'analyse du contenu de la base de données et des fichiers, la préparation du versement et le versement.

\clearpage

\insererImage{0.8}{illustrations/figure11.png}{Diagramme de la méthodologie de versement d’une application métier à décommissionner réalisé par Olkoa.}{figure11}


En conclusion de cette étude ressort globalement l’idée de créer une nouvelle fonctionnalité qui s'insèrerait dans la deuxième étape du processus décrit ci-dessus. Elle permettrait d’importer le CSV contenant des métadonnées supplémentaires extraites de \gls{SI} comme décrit ci-dessus dans \gls{Archifiltre}. Après l’import de l’arborescence extraite du \gls{SI}, tel qu’il se fait classiquement dans \gls{Archifiltre}, ce CSV pourrait alors être importé . Chaque ligne de ce CSV serait ensuite associée au document à laquelle elle se réfère. Ainsi, l’export \gls{Archifiltre} destiné à être importé dans ReSIP et à être versé à terme dans le \gls{SAE} \gls{VITAM}, serait enrichi afin d’y ajouter les métadonnées supplémentaires associées directement aux bons documents. Cet export enrichi vers ReSIP permettrait alors d’alimenter automatiquement les métadonnées de description de la balise <Content> et les métadonnées de gestion de la balise <management> pour chaque document, c’est à dire pour chaque \gls{unité d’archives} (ou <ArchiveUnit>).

\clearpage

\insererImage{0.4}{illustrations/figure12.png}{Description générale du besoin par Olkoa}{figure12}


Le besoin d’archivage des \gls{SI}, décrit par Olkoa ci-dessus spécifiquement pour être traité par \gls{Archifiltre}, est un besoin dont se sont saisis quelques acteurs comme l’Institut national de la propriété industrielle\footcite[p.10]{disic_mandat_2012} (INPI), le Ministère de la Justice ou encore le Commissariat à l'énergie atomique et aux énergies alternatives (CEA). Néanmoins, leur travail se concentre sur la mise en place d’un archivage en flux automatique à l’aide d’un connecteur. En s’intéressant à cette problématique, \gls{Archifiltre} cherche à proposer une solution alternative à la mise en place de ces solutions très coûteuses\footcite[p.12]{disic_mandat_2012} pour les services d'archives.


Suivant le but ainsi défini, un premier \gls{POC} (\textit{Proof of concept}) a été produit en septembre 2022.  Celui-ci avait pour but de confirmer rapidement la faisabilité de cette idée à l’équipe en interne et non pas d’être publié. Son utilisation était très limitée et la fonctionnalité avait alors été construite afin d’être testée précisément sur un CSV défini. Elle ne concernait également que l’import de métadonnées supplémentaires sans aborder l’association de ces dernières aux balises \gls{SEDA} adaptées lors de l’export vers ReSIP. Loin d’être abouti, ce premier POC a néanmoins permis de confirmer à l’équipe l’intérêt d’une telle fonctionnalité et lui a donné des premières pistes de réflexions pour établir les bases d’une future fonctionnalité plus développée. 


En conclusion, le besoin de faire évoluer les outils face à la problématique de la complexité des systèmes d’information à archiver se fait de plus en plus présent. L’équipe \gls{Archifiltre} a profité de ses expériences d’archivistes afin de se saisir du problème et de tenter de proposer une solution via son outil. Néanmoins, il est primordial que dans le développement de cette solution, les contraintes de la chaîne de traitement telle que définie dans la première partie de ce travail soient correctement prises en compte. De plus, le standard d’échange \gls{SEDA} constitue à lui seul une complexité inévitable pour l’équipe comme pour l’ensemble des archivistes numériques. Il est donc essentiel de le comprendre et de s’adapter à ses contraintes et à son fonctionnement afin de proposer une solution viable et facilitant l’emploi du \gls{SEDA} pour les archivistes. 
